\documentclass[acmsmall]{acmart}

\AtBeginDocument{%
	\providecommand\BibTeX{{%
			Bib\TeX}}}

\setcopyright{acmcopyright}
\copyrightyear{2022}
\acmYear{2022}
\acmDOI{XXXXXXX.XXXXXXX}


%%
%% These commands are for a JOURNAL article.
\acmJournal{JACM}
\acmVolume{37}
\acmNumber{4}
\acmArticle{111}
\acmMonth{8}

\begin{document}
	\title{Web Processing - Standardised GIS Analyses for Cable Route Planning}
	
	\author{Sebastian Heiden}
	\email{u38XXX@hs-harz.de}
	\affiliation{%
		\institution{Harz University of Applied Sciences}
		\streetaddress{Friedrichstrasse 57-59}
		\city{Wernigerode}
		\state{Saxony-Anhalt}
		\country{Germany}
		\postcode{38855}
	}
	
	
	\renewcommand{\shortauthors}{Heiden}
	
	\begin{abstract}
		add as last part
	\end{abstract}
	
	%%
	%% The code below is generated by the tool at http://dl.acm.org/ccs.cfm.
	%% Please copy and paste the code instead of the example below.
	%%
	\begin{CCSXML}
		<ccs2012>
		<concept>
		<concept_id>10010520.10010553.10010562</concept_id>
		<concept_desc>Computer systems organization~Embedded systems</concept_desc>
		<concept_significance>500</concept_significance>
		</concept>
		<concept>
		<concept_id>10010520.10010575.10010755</concept_id>
		<concept_desc>Computer systems organization~Redundancy</concept_desc>
		<concept_significance>300</concept_significance>
		</concept>
		<concept>
		<concept_id>10010520.10010553.10010554</concept_id>
		<concept_desc>Computer systems organization~Robotics</concept_desc>
		<concept_significance>100</concept_significance>
		</concept>
		<concept>
		<concept_id>10003033.10003083.10003095</concept_id>
		<concept_desc>Networks~Network reliability</concept_desc>
		<concept_significance>100</concept_significance>
		</concept>
		</ccs2012>
	\end{CCSXML}
	
	\ccsdesc[500]{Computer systems organization~Embedded systems}
	\ccsdesc[300]{Computer systems organization~Redundancy}
	\ccsdesc{Computer systems organization~Robotics}
	\ccsdesc[100]{Networks~Network reliability}
	
	%%
	%% Keywords. The author(s) should pick words that accurately describe
	%% the work being presented. Separate the keywords with commas.
	\keywords{datasets, neural networks, gaze detection, text tagging}
	
	\received{20 February 2007}
	\received[revised]{12 March 2009}
	\received[accepted]{5 June 2009}
	
	%%
	%% This command processes the author and affiliation and title
	%% information and builds the first part of the formatted document.
	\maketitle
	
	\section{Introduction}

	
	\section{Template Overview}

	
	\subsection{Template Styles}
	

	
	\subsection{Template Parameters}
	

	
	\section{Modifications}
	

	
	\section{Typefaces}
	

	
	\section{Title Information}
	

	
	\section{Authors and Affiliations}
	

	
	\section{Rights Information}
	

	
	\section{CCS Concepts and User-Defined Keywords}
	

	
	\section{Sectioning Commands}
	

	
	\section{Tables}
	

	
	\section{Math Equations}

	
	\subsection{Inline (In-text) Equations}

	
	\subsection{Display Equations}

	
	\section{Figures}
	
	
	\section{Citations and Bibliographies}
	

	\section{Acknowledgments}

\section{Multi-language papers}




\begin{acks}
	bla
\end{acks}

%%
%% The next two lines define the bibliography style to be used, and
%% the bibliography file.
\bibliographystyle{ACM-Reference-Format}
\bibliography{sample-base}


%%
%% If your work has an appendix, this is the place to put it.
\appendix

\section{Research Methods}

\subsection{Part One}



\subsection{Part Two}



\section{Online Resources}



\end{document}
\endinput
%%
%% End of file `sample-acmsmall.tex'.